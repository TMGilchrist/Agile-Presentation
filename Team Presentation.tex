\documentclass{beamer}
\usetheme{Boadilla}

\title{Agile Essay Research Presentation}
\date{04-10-2017}
\author{FruitBats}

%\usepackage{multibbl}
\usepackage{verbatim}

%\newbibliography{Mango}
%\newbibliography{Tomas}


\begin{document}

\begin{frame}
	\frametitle{Agile Essay Research}	
\end{frame}


%%----------------------------------------------------------------------------------------
%%		Mango's Slides
%%----------------------------------------------------------------------------------------

\begin{frame}
	\frametitle{Successfully Scaling the Agile Methodology}	
\end{frame}


\begin{frame}
	\frametitle{Criticism of Agile}
	
	Scalability is a common Criticism of the Agile methodology. 
	
	\vspace{5mm}
	
	Two Primary Scaling issues:	
	\begin{enumerate}
		\item Large scope, complex projects
		\item Large development teams
	\end{enumerate}	
\end{frame}


\begin{frame}
	\frametitle{Complex projects}
	
	\begin{enumerate}
		\item Requirement changes
		\item Lack of precise initial designs
		\item Mixing of Agile and Non-Agile teams
	\end{enumerate}
\end{frame}
	
	
\begin{frame}
	\frametitle{Large Teams}
	
	\begin{enumerate}
		\item As teams grow larger, communication becomes more difficult
		\item Lack of disciplined self-organisation
		\item Scrum meetings become increasingly impractical
	\end{enumerate}
\end{frame}


\begin{frame}
	\frametitle{Possible Solutions}
	
	\begin{enumerate}
		\item Scrum of Scrums	
		\item Lean Governance
		\item Multiple Backlogs
	\end{enumerate}
\end{frame}


\begin{frame}
	\frametitle{Scrum of Scrums}
	
	One of the most common solutions for implementing the Agile methodology in larger groups.
			
	\vspace{5mm}		
			
	\begin{enumerate}
		\item Development divided among agile teams, each team has an dedicated Ambassador
		\item The ambassadors from each team convene their own Scrum, communicating the progress of their own teams and discussing issues
	\end{enumerate}
	
\end{frame}


\begin{frame}
	\frametitle{Lean Development Governance}
	
	A framework for governing lean software development, created by Scott Ambler and Per Kroll of IBM.
	
	\vspace{5mm}
	
	Core principles of Lean Governance that can increase success when scaling agile development:		
	
	\vspace{3mm}
	
	\begin{enumerate}
		\item Risk-Based Milestones - Choosing high-value features to develop each sprint while aiming to reduce risk involved
		\item Align Team Structure with Architecture - Team structure will heavily influence program structure
		\item Staged Program Delivery - Each project in a program should choose their release dates and either meet them or skip them
	\end{enumerate}


\end{frame}


\begin{frame}
	\frametitle{Multiple Backlogs}
	
	A solution used by Nokia, utilising four levels of backlogs, along with a Scrum of Scrums.
	
	\vspace{5mm}
	
	Program Level:
	\begin{enumerate}
		\item Program Content Backlog - Top Level Requirements
		\item Program Backlogs - User stories for each release of the program
	\end{enumerate}

	\vspace{3mm}

	Team Level:
	\begin{enumerate}
		\item Scrum Team Backlogs - User stories for each individual team
		\item Sprint Backlogs - The tasks for each sprint
	\end{enumerate}

	\vspace{5mm}

	Relied on experienced managers to maintain long-term vision and ensure team cohesion.
\end{frame}


\begin{frame}
	\frametitle{Remaining Issues}
	
	Many scalability concerns that are not necessarily addressed by these suggestions.
	
	\vspace{5mm}
	
	In practice, some compromises may have to be made:
	\begin{enumerate}
		\item Lower expectations for change
		\item Create basic architecture frameworks to provide a base to work off
	\end{enumerate}
\end{frame}

%\nocite{Mango}{*}
%\bibliographystyle{Mango}{plain}
%\bibliography{Mango}{MangoReferences}{Frist bib}



%\bibliographystyle{plain}
%\bibliography{MangoReferences}


%%----------------------------------------------------------------------------------------
%%		Tomas' Slides
%%----------------------------------------------------------------------------------------


\begin{frame}
 	\frametitle{Tomas}
\end{frame}

\begin{frame}
\frametitle{How to Start?}
First of all I'm going to talk about the most common ways of 
making a project more pleasant to work on as well as more
precise. For this part Mike Cohn's book \textit{Agile estimating and planning}
will help a lot. \cite{cohn2005book} His book does not only include 
explanations, but also pretty funny examples, which I may include.
\end{frame}

\begin{frame}
\frametitle{What's Next?}
While writing and discussing about importance of planning and estimating I will
briefly talk about some estimating techniques \cite{molokken2007combining}
\cite{tamrakar2012does} and different plannings, which
are release planning and iteration planning \cite{cohn2005book}.
\end{frame}

\begin{frame}
\frametitle{Agile Game Development}
Continuing planning and estimating topics I will try and talk 
about that from game development side. Therefore, I'll include
some experiments that show why a specific technique is more
precise than any other \cite{haugen2006empirical}\cite{grapenthin2016supporting}.
\end{frame}

\begin{frame}
\frametitle{Planning Poker}
I will eventually mention planning poker not only as a 
precise estimating technique, but also as a very exciting
and emotionally engaging \cite{yip2007hands}. Thus, it will 
increase the engagement level in the release planning process.
Planning poker also solves some problems, which are quite an
issue if one doesn't know how to deal with it \cite{grenning2002planning}.
\end{frame}




%%----------------------------------------------------------------------------------------
%%		Ryan's Slides
%%----------------------------------------------------------------------------------------

	\begin{frame}
		frametitle{Ryan}
	\end{frame}

	\small
	\begin{frame}
	
	\textbf{Agile software testing in a large-scale project} 
	
	Focuses on implementing agile testing for a large government military project. It shows a cut by an order of magnitude the time
	required to fix defects, defect longevity, and defect-management overhead.\\
	The agile process required a new mind-set at personal and organizational levels.
	\end{frame}
	\begin{frame}
	\textbf{Communication between Developers and Testers in Distributed Continuous Agile Testing}
	
	Four main types of communication between tester and developer;
	\begin{itemize}
		\item Handover through issue tracker system
		\item Formal meetings
		\item Written communication
		\item Coordination by mutual adjustment
	\end{itemize}
	Early participation of the testers is very important to the success of the handover between testers and developers accomplished by attending planning, standups and review meetings. \\
	Communication is not sufficiently effective through written communication and it must be augmented by informal communication.	
	\end{frame}
	\begin{frame}

	\textbf{Games User Research (GUR) for Indie Studios }	
	
	For playtesting you must use your relevant target audience for the findings to confidently be applied to the game.
	Using a persona to match correct testers will make it easier to find the ones matching your target audience.\\
	Having accurate data for not only the gameplay but also user facial expressions sound and actions outside of the game lets you more accurately messure what is enjoyable.\\
	Developers being present at playtests to witness the players experience makes the devs more motivated to fix issues imediately and note what the player found enjoyable to provide similar experiences in the future
	\end{frame}
	\begin{frame}
	\textbf{Using prototypes in early pervasive game development}
	
	This disscussed using "real" players and professional test players.
	Players who belong to the target group of the game usually provide more relevant data and are useful for understanding the players’ attitudes, opinions, and behavior. 
	Using professional test players (e.g., colleagues) for testing enables faster iteration, and is also beneficial when new ideas are needed. If the prototype is very incomplete, it can be difficult for outsiders to understand, so it is useful to have both kinds of test players in the same project.
	\end{frame}
	\begin{frame}
	\textbf{I have no words and I must design}
	
	Game design is ultimately a process of iterative refinement, continuous adjustment during testing, until, budget, schedule and management willing, we have a polished product that does indeed work beautifully, wonderfully, superbly.\\
	But your chances of getting that beautiful, wonderful, superb game will be much higher if you intentionality begin by thinking about the experiences you want your players to have, understand what makes a game, and understand what pleasures people find in them.
	\end{frame}


%%----------------------------------------------------------------------------------------
%%		Louis' Slides
%%----------------------------------------------------------------------------------------



\begin{frame}
	frametitle{Louis}
\end{frame}

\begin{frame}
	\frametitle{Question}
	\tableofcontents
\end{frame}

\section{Can learning be incorporated into production?}
\begin{frame}
	\frametitle{Can learning be incorporated into production?}
	\begin{itemize}
		\item Yes -- but not without its flaws
		\begin{itemize}
			\item 'Ability to learn while working' itself is a vital quality to employers (Jussi Kasurinen et al., 2017)
			\item Requires motivation
		\end{itemize}
	\end{itemize}
	\begin{itemize}
		\item Common issues that arise in group work include
		\begin{itemize}
			\item Pressure
			\item Scope
			\item Issues in task setting
		\end{itemize}
	\end{itemize}
\end{frame}

\section{How fit is agile for learning?}
\begin{frame}
	\frametitle{How fit is agile for learning?}
	\begin{itemize}
		\item Focus on communication and interactions enables peer learning
		\item Iteration focuses on progress rather than goal
		\item Active task setting promotes autonomy
		\begin{itemize}
			\item Autonomy considered a key factor in improving motivation (E. Deci and R. Ryan, 2012)
		\end{itemize}
	\end{itemize}
\end{frame}

\section{Could it be better?}
\begin{frame}
	\frametitle{Could it be better?}
	\begin{itemize}
		\item Consider adding 'what I want to learn today' to standups
		\begin{itemize}
			\item Review and self-reflection has proven educational benefits (S. Edmunds \& G. Brown, 2010)
		\end{itemize}
		\item Promote 'working in pairs'
		\begin{itemize}
			\item Some game companies suggest this helps learning and troubleshooting (M. Tran \& R. Biddle, 2008)
			\item Promotes active orientation--learning for teaching--which benefits learning (C. Benware and E. Deci, 1984)
		\end{itemize}
	\end{itemize}
\end{frame}




%%----------------------------------------------------------------------------------------
%%		Beren's Slides
%%----------------------------------------------------------------------------------------




\nocite{*}
\bibliography{AllReferences}
\bibliographystyle{plain}

\end{document}